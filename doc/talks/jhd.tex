\documentclass[xcolor=dvipsnames]{beamer}
\usepackage[utf8]{inputenc}
\usepackage[T1]{fontenc}
\usepackage{fixltx2e}
\usepackage{graphicx}
\usepackage{longtable}
\usepackage{float}
\usepackage{wrapfig}
\usepackage{soul}
\usepackage{textcomp}
\usepackage{marvosym}
\usepackage{wasysym}
\usepackage{latexsym}
\usepackage{amssymb}
\usepackage{hyperref}
\usepackage{cancel}
\usepackage[percent]{overpic}
\usepackage{listings}
\usepackage{color}
\lstset{ %
  language=Python,
  basicstyle=\ttfamily\tiny,
  emphstyle=\color{red},
  keywordstyle=\color{black}\bfseries,
  identifierstyle=\color{DarkOrchid}\ttfamily,
  commentstyle=\color{Brown}\rmfamily\itshape,
  stringstyle=\color{blue}\slshape,
  showstringspaces=false,
  frame=single,                   % adds a frame around the code
}

\setbeamertemplate{navigation symbols}{}
\useoutertheme{infolines}
\usecolortheme[named=violet]{structure}
\setbeamertemplate{items}[circle]

\newcommand{\code}[1]{\texttt{#1}}

\DeclareGraphicsExtensions{.pdf,.png,.jpg}
\title[Job Harness]{LCATR Job Harness Design and Status}
\author{Brett Viren}
\institute[BNL]
{
  Physics Department

  \includegraphics[height=1.5cm]{bnl-logo}
}

\date{\today}

\definecolor{rootpink}{RGB}{255,0,255}
\definecolor{macroyellow}{RGB}{255,215,0}

\hypersetup{
  pdfkeywords={},
  pdfsubject={},
  pdfcreator={pdflatex, beamer and emacs the digital blood, sweat and tears}}

\begin{document}

\maketitle

\begin{frame}
\frametitle{Outline}
\tableofcontents
\end{frame}

\section{Overview, Terminology and Context}

\begin{frame}
  \frametitle{Purpose of the Job Harness}

  The job harness:
  \begin{itemize}
  \item Runs a \textbf{job} that implements a portion or the whole of
    some CCD/RTM \textbf{test}.  For example jobs that implement:
    \begin{itemize}
    \item all of or a portion of one station's measurements
    \item an analysis of a station's output or that of another analysis
    \item reformatting of high-bit FITS images to thumbnails for web display
    \item combining or comparisons of results across a number of tests
    \end{itemize}
  \item Specify and capture the software version, input parameters and
    run-time environment of a job.
  \item Provides a uniform user interface to the operators running jobs.
  \item Assure that any required output from other tests is available for
    input to the current one.
  \item Assure output results are validated and secured in an
    organized archive.
  \end{itemize}
\end{frame}

\begin{frame}
  \frametitle{Terminology}
  \begin{description}
  \item[Unit] a CCD or an RTM
  \item[Test] some evaluation of a unit that produces data.
    \begin{description}
      \item[Station] hardware/software/human based measurement
      \item[Analysis] pure software derived result
    \end{description}
  \item[Job] a software-only, portion or the whole of a test that can
    be run atomically and once started run non-interactively
  \item[Program] an executable from the test software that implements the job
  \item[Archive] a site-central file system storing all results in an
    organized directory hierarchy
  \item[Stage] a portion of the archive copied to the test host
    computer and which accepts output of a running job.
  \end{description}
\end{frame}

\begin{frame}
  \frametitle{High Level Context and Roles for a Test Site}
  \includegraphics[width=\textwidth]{highlevel}

  \begin{itemize}
  \item Software developers tag their \code{git} repo, tell software mgr the tag
  \item Software managers install tagged version and create \textit{modulefile}
  \item Operator runs software jobs via the \textit{job harness}
  \item Job harness archives results and interfaces with LIMS
  \end{itemize}

\end{frame}

\begin{frame}
  \frametitle{Multiple Test Sites}
  \begin{center}
    Sites operate as symmetric peers with uncoupled replication.    
  \end{center}


  Every site has its own local:
  \begin{description}
  \item[LIMS] web application and database
    \begin{itemize}
    \item Site database is writable only by site LIMS web application instance
    \item Every site copies (MySQL replication) other sites' database in read-only manner
    \end{itemize}
  \item[Archive] disk area holding all result files produced at the site
    \begin{itemize}
    \item Writable only by local job harness runs.
    \item Every other site may copy (rsync) other sites' archive
    \end{itemize}
  \item[Code repositories] for LIMS app, job harness related and test software itself
    \begin{itemize}
    \item Local Git clones with periodic pulls
    \end{itemize}
  \end{description}

  Can include SLAC as a (non producing) peer to deliver data in ``real time''.
\end{frame}

\section{Design}

\begin{frame}
 \tableofcontents[currentsection,hideothersubsections]
\end{frame}



\subsection{Job Execution Cycle}

\begin{frame}
  \frametitle{Job Execution Cycle}
  Harness executes through these steps, in order:
  \begin{description}
  \item[configuration:] input parameters and context-specific static configuration
  \item[registration:] parameters registered with LIMS, unique job
    ID allocated, dependencies resolved.
    % fixme: fix this label in the text
  \item[environment:] run-time environment configured via the job's \textit{modulefile}
  \item[staging:] any input file dependencies copied to the local stage
  \item[running:] the test software is run in two steps:
    \begin{enumerate}
    \item data production (primary part of the test software) program is run
    \item validation and production of \textit{metadata} and
      \textit{summary result} files
    \end{enumerate}
  \item[archiving:] results are copied to archive
  \item[termination:] optional clean up and final termination
  \end{description}
  LIMS is notified of the completion of these steps.
\end{frame}

\subsection{Job Harness Configuration}

\begin{frame}
  \frametitle{Configuration Mechanisms}
  Configuration is done through a cascade of sources:

  \begin{enumerate}
  \item Local shell account environment.
  \item Configuration file defaults.
  \item Configuration file sections based on existing parameter
    settings.
  \item User interface (command line) parameter settings.
  \end{enumerate}

  Allows for static, context-specific (site, computer, job type,
  account) defaults and dynamic, job/run-specific (CCD ID, job
  version) settings.
  
\end{frame}
\begin{frame}
  \frametitle{Configuration Parameters}
  Main parameters:
  \begin{description}
  \item[lims\_port] specify LIMS port (different access for
    production, testing and development instances)
  \item[job] the (canonical) name for a job.
  \item[version] job version string (Git tag)
  \item[operator] the username of the account invoking the harness
  \item[archive\_root] absolute path to the root of the archive file system.
  \item[archive\_user] user that has SSH access to the machine providing
  \item[archive\_host] name of the computer providing the
    archive file system.
  \item[stage\_root] the absolute path to the root of the stage file
    system local to the computer running the job.
  \item[...] others

  \end{description}

\end{frame}


\subsection{Job Environment}

\begin{frame}
  \frametitle{Specifying a Job and its Environment Module}
  
  A job's software is uniquely identified by its:

  \begin{description}
  \item[name] A canonical name for the job
  \item[version] A version string
  \end{description}

  These are associated with:

  \begin{itemize}
  \item A specific commit of the software in Git
  \item An installed instance of this software
  \item Specific executable programs provided by the installation
  \item Any special environment settings needed by the software
  \item Environment in support of the Job Harness
  \end{itemize}

  via an \textit{Environment Module} file (\textit{modulefile}).

  \vspace{5mm}

  \textbf{NB:} This association relies on the site's software
  manager(s) to correctly construct the \textit{modulefile}.  Where
  possible, installation and \textit{modulefile} generation will be
  automated.

\end{frame}

\begin{frame}
  \frametitle{Environment Modules}

  What are Environment Modules?

  \begin{center}
    \url{http://modules.sf.net/}
  \end{center}
    
  \begin{itemize}
  \item describe run-time environment in shell-neutral language
    \begin{itemize}
    \item write a \textit{modulefile} (in TCL) for each version of each job type
    \end{itemize}
  \item normally used to modify a user's shell environment to
    introduce new packages built from source
    \begin{itemize}
    \item the job harness loads the \textit{modulefile} from \textbf{Python}
    \end{itemize}
  \item allow simple dependencies resolution between versioned modules
  \item assure uniform behavior via a common library of functions
  \end{itemize}

  Modules are used to manage user environment at some major facilities
  (eg. RACF, PDSF@NERSC)

\end{frame}

\subsection{File System Organization and Preparation}

\begin{frame}
  \frametitle{Archive/Stage Organization and Preparation}
  {\footnotesize
    \begin{center}
      \code{<root>/<unit\_type>/<unit\_id>/<job\_name>/<job\_version>/<job\_id>/}
    \end{center}}
  \begin{itemize}
  \item Every invocation of a job produces output in a unique location.
  \item The path is rooted at:
    \begin{description}
    \item[\code{<archive\_root>}] on the site-central archive file system
    \item[\code{<stage\_root>}] on the local file system hosting the job
    \end{description}
  \item Job harness preps the local stage at the start of the job
  \item Results are coped to the archive at the end.
  \end{itemize}

\end{frame}

\subsection{Job Data Production and Validation Steps}

\begin{frame}
  \frametitle{Running the actual test software}

  Two step process:
  \begin{description}
  \item[Production] implements the job's primary intention
    \begin{itemize}
    \item Typically a thin script wrapping one or more programs
    \item Produces output files in whatever form is natural
    \end{itemize}
  \item[Validation] validates and formats the results
    \begin{itemize}
    \item Runs any job-specific validation code
    \item Must produce \textit{metadata} and \textit{results summary}
      files in standardized format/schema
    \item Job harness runs associated validation code on these standard
      files.
    \end{itemize}
  \end{description}

  For both, standard out/err is logged and a zero return code is used to
  indicate success to the calling harness.

\end{frame}

\subsection{Interaction With LIMS}

\begin{frame}
  \frametitle{Interface with LIMS}
  \begin{center}
    \includegraphics[width=0.8\textwidth]{lims-interaction}    
  \end{center}
  \begin{itemize}
  \item LIMS web app gives big-picture and per unit status to operators
  \item Job harness instances run by operator, updates status with LIMS
  \item Ingest process parses standardized \textit{metadata} and
    \textit{results summary} files into LIMS
  \end{itemize}
\end{frame}

\begin{frame}
  \frametitle{Use LIMS for: Job ID Allocation, Dependency Resolution and Status Updates}

  \begin{description}
  \item[Job ID] Allocated through LIMS by registering job input
    parameters.  ID is unique to the instance of LIMS and used to
    track the job from them on.
  \item[Dependencies] LIMS also returns information about any other
    jobs on which the current job depends.  This information is used
    to resolve archive paths so they can be copied to the local stage
    and be available to the job.
  \item[Status] The Job Harness registers status updates with LIMS:
    \code{registered}, \code{configured}, \code{staged},
    \code{produced}, \code{validated}, \code{archived}, \code{purged}
    (and finally the ingest process records \code{ingested}).
  \end{description}

\end{frame}

\section{Status}

\begin{frame}
 \tableofcontents[currentsection,hideothersubsections]
\end{frame}

\begin{frame}
  \frametitle{Status of this Stuff}
  \begin{description}
  \item[schema] \textit{metadata} and \textit{results summary} files
    \begin{itemize}
    \item format chosen (FITS) and schema implementation (based on
      pyfits) started
    \item one prototype result implemented
    \item draft tech note and reference docs (Sphinx) online
    \end{itemize}
  \item[harness] basic design and interaction with LIMS understood
    \begin{itemize}
    \item harness code partially implemented and tested on Linux
    \item MS Windows test (virtual) machine installed and ready
    \item draft tech note online
    \end{itemize}
  \item[modulefiles] selected to specify/provide job environment
    \begin{itemize}
    \item brief evaluation of roll-our-own Python implementation
    \item initial prototype \textit{modulefiles} developed
    \item some utility TCL functions started but still working on
      minimizing required boilerplate
    \end{itemize}
  \end{description}
\end{frame}

\begin{frame}
  \frametitle{Some Links}

  \begin{description}
  \item[Online] code, tech notes, reference docs and my working area:\\
    \url{http://www.phy.bnl.gov/~bviren/lsst/lcatr/}
  \item[Git] repository collections:
    \url{https://git.racf.bnl.gov/astro/cgit/lcatr/}
    \url{http://dev.lsstcorp.org/cgit/LSST/Camera/lcatr/}
  \end{description}

  \vspace{1cm}
  \footnotesize
  Notes:
  \begin{itemize}
  \item In each look for \code{schema/}, \code{harness/} and
    \code{modulefiles} subdirs
  \item RACF Git currently requires RACF credentials.  LSST Corp's are
    mirrors.
  \end{itemize}
\end{frame}

\section{Some Issues for Discussion}

\begin{frame}
 \tableofcontents[currentsection,hideothersubsections]
\end{frame}

\begin{frame}
  \frametitle{Data Delivery to SLAC}

  \begin{itemize}
  \item Bare minimum effort: tar up the archives and dump the MySQL
    databases at end of all testing.
  \item Almost trivial extra: synchronize SLAC as if it were a testing
    peer site.
  \item Going further: will any data conversion be needed?
  \end{itemize}

  Note: the \textbf{content} of the data needs a separate discussion
  (in the ``algorithm'' or ``data products'' groups?)

\end{frame}

\begin{frame}
  \frametitle{Use of LIMS / Job Harness beyond CCD testing?}

  Can we use all or some of LIMS / Job Harness system beyond
  sensor-level testing?  Maybe.  No objections here, but:

  \begin{itemize}
  \item Focus is now on the more immediate CCD/RTM testing needs.
  \item Some generality is included in the design naturally, but no
    attempt is made to guess at how to accommodate full-camera testing.
    \begin{itemize}
    \item Requirements for full camera testing are unknown (by me at least)
    \item Effort not identified to increase the scope at this time.
    \end{itemize}
  \item What about slice-tests that may require CCS?  My understanding
    is CCS would run as run as (or be couched into) a job running
    under the harness.
  \end{itemize}

\end{frame}

\end{document}


