\documentclass[xcolor=dvipsnames]{beamer}
\usepackage[utf8]{inputenc}
\usepackage[T1]{fontenc}
\usepackage{fixltx2e}
\usepackage{graphicx}
\usepackage{longtable}
\usepackage{float}
\usepackage{wrapfig}
\usepackage{soul}
\usepackage{textcomp}
\usepackage{marvosym}
\usepackage{wasysym}
\usepackage{latexsym}
\usepackage{amssymb}
\usepackage{hyperref}
\usepackage{cancel}
\usepackage[percent]{overpic}
\usepackage{listings}
\usepackage{color}
\lstset{ %
  language=Python,
  basicstyle=\ttfamily\tiny,
  emphstyle=\color{red},
  keywordstyle=\color{black}\bfseries,
  identifierstyle=\color{DarkOrchid}\ttfamily,
  commentstyle=\color{Brown}\rmfamily\itshape,
  stringstyle=\color{blue}\slshape,
  showstringspaces=false,
  frame=single,                   % adds a frame around the code
}

\setbeamertemplate{navigation symbols}{}
\useoutertheme{infolines}
\usecolortheme[named=violet]{structure}
\setbeamertemplate{items}[circle]

\newcommand{\code}[1]{\texttt{#1}}

\DeclareGraphicsExtensions{.pdf,.png,.jpg}
\title[Job Harness]{LCATR Job Harness Design and Status}
\author{Brett Viren}
\institute[BNL]
{
  Physics Department

  \includegraphics[height=1.5cm]{bnl-logo}
}

\date{\today}

\definecolor{rootpink}{RGB}{255,0,255}
\definecolor{macroyellow}{RGB}{255,215,0}

\hypersetup{
  pdfkeywords={},
  pdfsubject={},
  pdfcreator={pdflatex, beamer and emacs the digital blood, sweat and tears}}

\begin{document}

\maketitle

\begin{frame}
\frametitle{Outline}
\tableofcontents
\end{frame}

\section{Overview, Terminology and Context}

\begin{frame}
  \frametitle{Purpose of the Job Harness}

  The job harness:
  \begin{itemize}
  \item Runs a job that implements a portion or the whole of some
    CCD/RTM ``test'' (a station run or an analysis).
  \item Specify and capture software version, input parameters and
    run-time environment.
  \item Provides a uniform interface to the operators running tests.
  \item Assure any required output from other tests is available for
    input to the current one.
  \item Assure output results are validated and secured in an
    organized archive.
  \end{itemize}
\end{frame}

\begin{frame}
  \frametitle{Terminology}
  \begin{description}
  \item[Unit] CCD a or an RTM
  \item[Test] Some evaluation of a unit that produces data.
    \begin{description}
      \item[Station] hardware/software/human based measurement test
      \item[Analysis] pure software based test
    \end{description}
  \item[Job] a software portion or sub-part of a test that can be run atomically
  \item[Program] an executable from the test software that implements a job
  \item[Archive] a site-central file system storing all results in an organized manner
  \item[Stage] a portion of the archive copied to the test host
    computer and accepting output of a running job.
  \end{description}
\end{frame}

\begin{frame}
  \frametitle{High Level Context and Roles for a Test Site}
  \includegraphics[width=\textwidth]{highlevel}

  \begin{itemize}
  \item Software developers tag their \code{git} repo, tell software mgr the tag
  \item Software managers install tagged version and create \textit{modulefile}
  \item Operator runs software jobs via the \textit{job harness}
  \item Job harness archives results and interfaces with LIMS
  \end{itemize}

\end{frame}

\begin{frame}
  \frametitle{Interface with LIMS}
  \begin{center}
    \includegraphics[width=0.8\textwidth]{lims-interaction}    
  \end{center}
  \begin{itemize}
  \item LIMS web app big-picture and per unit status to operators
  \item Job harness instances run by operator, updates status with LIMS
  \item Ingest process parses standardized \textit{metadata} and
    \textit{results summary} files into LIMS
  \end{itemize}
\end{frame}

\begin{frame}
  \frametitle{Multiple Test Sites}
  \begin{center}
    Sites operate as symmetric peers with uncoupled replication.    
  \end{center}


  Every site has its own local:
  \begin{description}
  \item[LIMS] web application and database
    \begin{itemize}
    \item Site database is writable only by site LIMS web application instance
    \item Every site copies (MySQL replication) other sites' database in read-only manner
    \end{itemize}
  \item[Archive] directory structure holding results
    \begin{itemize}
    \item Writable only by local job harness.
    \item Every site copies (rsync) other sites' archive in read-only manner
    \end{itemize}
  \item[Code repositories] for LIMS, job harness, test software
    \begin{itemize}
    \item Local clones periodically pull updates.
    \end{itemize}
  \end{description}

  Can include SLAC as a (non producing) peer to deliver data in ``real time''.
\end{frame}

\section{Design}

\begin{frame}
 \tableofcontents[currentsection,hideothersubsections]
\end{frame}



\subsection{Job Execution Cycle}

\begin{frame}
  \frametitle{Job Execution Cycle}
  Harness executes through these steps, in order:
  \begin{description}
  \item[configuration] input parameters and context-specific static configuration
  \item[registration] parameters registered with LIMS, unique job
    ID allocated, dependencies resolved.
    % fixme: fix this label in the text
  \item[environment] run-time environment configured via the job's \textit{modulefile}
  \item[staging] any input file dependencies copied to the local stage
  \item[running] the test software is run in two steps
    \begin{enumerate}
    \item data production program run
    \item validation and \textit{metadata} and \textit{summary result}
      files produced
    \end{enumerate}
  \item[archiving] results are copied to archive
  \item[termination] optional clean up and final termination
  \end{description}
  LIMS is notified of the completion of these steps.
\end{frame}

\subsection{Job Harness Configuration}

\begin{frame}
  \frametitle{Configuration Mechanisms}
  Configuration is done through a cascade of sources:

  \begin{enumerate}
  \item Local shell account environment.
  \item Configuration file defaults.
  \item Configuration file sections based on existing parameter
    settings.
  \item User interface (command line) parameter settings.
  \end{enumerate}

  Allows for static, context-specific (site, computer, job type,
  account) defaults and dynamic, job/run-specific (CCD ID, job
  version) settings.
  
\end{frame}
\begin{frame}
  \frametitle{Configuration Parameters}
  Main parameters:
  \begin{description}
  \item[lims\_port] specify LIMS port (allows for production, testing, development)
  \item[job] the (canonical) name for a job.
  \item[version] job version string (Git tag)
  \item[operator] the username of the account invoking the harness
  \item[archive\_root] absolute path to the root of the archive file system.
  \item[archive\_user] the name of the user that has SSH access to the machine providing
  \item[archive\_host] the host name of the computer providing the
    archive file system.
  \item[stage\_root] the absolute path to the root of the stage file
    system local to the computer running the job.
  \end{description}
  Other settings are available.
\end{frame}


\subsection{Job Environment}

\begin{frame}
  \frametitle{Specifying a Job and its Environment Module}
  
  A job is uniquely identified by its:

  \begin{description}
  \item[name] A canonical name for the job
  \item[version] A version string
  \end{description}

  These are associated with:

  \begin{itemize}
  \item A specific commit of the test software in Git
  \item An installed instance of this software
  \item Specific executable programs provided by the installation
  \item Any special environment settings needed by the software
  \item Environment in support of the Job Harness
  \end{itemize}

  via an \textit{Environment Module}.

  \textbf{NB:} This association relies on the site's software
  manager(s) to correctly construct the \textit{modulefile}.  Where
  possible, installation and \textit{modulefile} generation will be
  automated.

\end{frame}

\begin{frame}
  \frametitle{Environment Modules}

  What are Environment Modules\footnote{\url{http://modules.sf.net/}}?
    
  \begin{itemize}
  \item describe run-time environment in shell-neutral language
    \begin{itemize}
    \item write a \textit{modulefile} (in TCL) for each version of each job type
    \end{itemize}
  \item normally used to modify a user's shell environment to introduce packages
    \begin{itemize}
    \item the job harness loads \textit{modulefile} from \textbf{Python}
    \end{itemize}
  \item allow simple dependencies resolution between versioned modules
  \item assure uniform behavior via a shared library of functions
  \end{itemize}

  Modules are used to manage user environment at some major facilities
  (eg. RACF, PDSF@NERSC)

\end{frame}

\subsection{File System Organization and Preparation}

\begin{frame}
  \frametitle{Archive/Stage Organization and Preparation}
  {\footnotesize
    \begin{center}
      \code{<root>/<unit\_type>/<unit\_id>/<job\_name>/<job\_version>/<job\_id>/}
    \end{center}}
  \begin{itemize}
  \item Every invocation of a job produces output in a unique location.
  \item The path is rooted at:
    \begin{description}
    \item[\code{<archive\_root>}] on the site-central archive file system
    \item[\code{<stage\_root>}] on the local file system hosting the job
    \end{description}
  \item Job harness preps the local stage at the start of the job
  \item Results are coped to the archive at the end.
  \end{itemize}

\end{frame}

\subsection{Job Data Production and Validation Steps}

\begin{frame}
\end{frame}

\subsection{Interaction With LIMS}

\begin{frame}
\end{frame}

\section{Status}

\begin{frame}
 \tableofcontents[currentsection,hideothersubsections]
\end{frame}

\section{Open Issues}

\begin{frame}
 \tableofcontents[currentsection,hideothersubsections]
\end{frame}

\begin{frame}
  \frametitle{Data Delivery to SLAC}

  \begin{itemize}
  \item Bare minimum effort: tar up the archives and dump the MySQL
    databases at end of all testing.
  \item Almost trivial extra: synchronize SLAC as if it were a testing
    peer site.
  \item Going further: will any data conversion be needed?
  \end{itemize}

  Note: the \textbf{content} of the data needs a separate discussion
  (in the ``algorithm'' or ``data products'' groups?)

\end{frame}

\begin{frame}
  \frametitle{Use this for Full-Camera Tests?}

  Can we use all or some of LIMS / Job Harness system for full-camera
  testing?  Maybe.  No objections here, but:

  \begin{itemize}
  \item Focus is now on the more immediate CCD/RTM testing needs.
  \item Some generality is included in the design naturally, but no
    attempt is made to guess at how to accommodate full-camera testing.
    \begin{itemize}
    \item Requirements for full camera testing are unknown (by me at least)
    \item Effort not identified to increase the scope at this time.
    \end{itemize}
    
  \end{itemize}

\end{frame}

\end{document}


